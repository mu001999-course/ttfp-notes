\documentclass[UTF8]{article}
\usepackage{ctex}
\usepackage{ulem}
\usepackage{dsfont}
\usepackage{amssymb}
\usepackage{amsmath}
\usepackage{graphicx}
\newtheorem{thm}{定义}[section]
\newtheorem{notation}[thm]{记号}
\newtheorem{lemma}[thm]{引理}

\makeatletter
\newcommand{\rmnum}[1]{\romannumeral #1}
\newcommand{\Rmnum}[1]{\expandafter\@slowromancap\romannumeral #1@}
\makeatother
\newcommand{\dperp}{\perp\!\!\!\perp}

\title{14 $\lambda{\rm D}$中的数字与算术\\Numbers and arithmetic in $\lambda{\rm D}$\\[2ex]\begin{large}读书笔记\end{large}}
\author{许博}
\date{}

\begin{document}
\maketitle
	\section{用于自然数的皮亚诺公理}
	\noindent
	皮亚诺假设存在一个集合$\mathbb{N}$,一个特殊的成员0,以及一个由$\mathbb{N}$到$\mathbb{N}$的函数$s$。所以在$\mathbb{N}$中,我们有成员0,$s(0)$,$s(s(0))$等,表示0,1,2等。
	
		之后,皮亚诺通过添加公理,使得这些形式化的数字行为符合预期。为了保证函数$s$一定产生新的数字,皮亚诺添加了两条公理:
		
		$ax{-}nat_1:\ \forall_{x\in\mathbb{N}}(s(x)\not=0)$
		
		$ax{-}nat_2:\ \forall_{x,y\in\mathbb{N}}(s(x)=s(y)\Rightarrow x=y)$
		
		公理$ax{-}nat_2$表示$s$是一个内射的函数,而$ax{-}nat_1$隐含了$s$不是满射的。这两条公理决定了不同层数$s$的自然数不相同。
		
		除此之外,皮亚诺还添加了另一条公理,以通过数学归纳法确定所有自然数的性质:
		
		$ax{-}nat_3:\ (P0\land\forall x:\mathbb{N}.(Px\Rightarrow P(sx)))\Rightarrow\forall x:\mathbb{N}.Px$
		
		\begin{lemma} 对于所有$n\in\mathbb{N}:n=0\lor\exists_{m\in\mathbb{N}}(n=s(m))$
		\end{lemma}
\end{document}
