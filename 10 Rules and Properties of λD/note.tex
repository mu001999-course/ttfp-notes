\documentclass[UTF8]{article}
\usepackage{ctex}
\usepackage{ulem}
\usepackage{amssymb}
\usepackage{amsmath}
\usepackage{graphicx}
\newtheorem{thm}{定义}[section]
\newtheorem{notation}[thm]{记号}
\newtheorem{lemma}[thm]{引理}

\makeatletter
\newcommand{\rmnum}[1]{\romannumeral #1}
\newcommand{\Rmnum}[1]{\expandafter\@slowromancap\romannumeral #1@}
\makeatother

\title{10 $\lambda{\rm D}$的规则与性质\\Rules and properties of $\lambda{\rm D}$\\[2ex]\begin{large}读书笔记\end{large}}
\author{许博}
\date{}

\begin{document}
\maketitle
	\section{描述性定义与基本(primitive)定义}
	\noindent
	上一章中,我们定义了基于描述性定义的系统$\lambda{\rm D_0}$,将定义作为“一等公民”扩展了$\lambda{\rm C}$。其中“描述性”指的是每一个被定义的常量都与一个确切的定义实体(definiens)相关联,并且被给定一个形式化的描述以表示这个常量代表了什么。
	
		而在通常的数学和逻辑中,还需要表示所谓的基本概念(primitive  notions),用于结合公理和公理性概念。
		
		基本定义中所引入的常量与描述性定义中相异的是,它不与一个描述性的表达式相关联,只提供它的类型,而没有更多的限制。因此,基本的常量也不能展开。
		
		一个基本定义可以被看作是公理性的引入逻辑或者数学中一个假设存在但不能构造出来的对象。同样可以在$\lambda{\rm C}$用于假设成立但无法证明的一个公理,比如在章节7.4中提到的DN公理。
		
		描述性定义和基本定义的本质区别在于定义实体(definiens)的存在与否,而它们一致的地方在于:两者中常量的行为相似,都是它们参数列表的实例化;以及两者中的常量都关联于一个类型。
		
		因为这些相似性,我们选择将这两者的引入过程合二为一。也意味着我们将扩展“定义”到基本或者公理性的名字。
		
		以基本定义扩展$\lambda{\rm D_0}$后,得到的系统称为$\lambda{\rm D}$。系统$\lambda{\rm D}$是我们最终的形式化机器,非常适合形式化大量的数学,其中一个好处是,形式化的过程强制形式化的内容得到完全的验证。
\end{document}
