\documentclass[UTF8]{article}
\usepackage{ctex}
\usepackage{ulem}
\usepackage{amssymb}
\usepackage{amsmath}
\usepackage{graphicx}
\newtheorem{thm}{定义}[section]
\newtheorem{notation}[thm]{记号}
\newtheorem{lemma}[thm]{引理}

\makeatletter
\newcommand{\rmnum}[1]{\romannumeral #1}
\newcommand{\Rmnum}[1]{\expandafter\@slowromancap\romannumeral #1@}
\makeatother

\title{8 定义\\Definitions\\[2ex]\begin{large}读书笔记\end{large}}
\author{许博}
\date{}

\begin{document}
\maketitle
	\section{定义的本质(nature)}
	\noindent
	在逻辑和数学课本中,定义是必要的。因此我们将为$\lambda{\rm C}$添加含有定义的扩展,得到的推导系统记为$\lambda{\rm D}$,将在第十章中描述。而一个简单的前身记为$\lambda{\rm D_0}$,将在第九章中描述。本章将讨论定义的本质特征,以及如何形式化它们。
	
		引入定义的主要原因是为了表示并突出(highlight)有用的概念。逻辑和数学都基于某些概念,其中大部分都是其它概念的复合。通过给它们指定名称来挑选值得注意的概念非常方便。
		
		如“一个矩形是一个有四个直角的四边形”,值得注意的概念是“一个有四个直角的四边形”,而我们给它指定名称“矩形”。其中“四边形”以及“直角”的解释假定已经在之前给出。
		
		引入的名字不仅可以使用自然语言中的单词,也可以新发明单词或者符号,比如 c 或 $D_n$。而像 c 或$D_n$多半用于临时使用,但它们和更永久(permanent)的名字(如“矩形”)没有本质区别。所有名字的一个重要特征就是可以使重复引用了这些名字关联的对象或概念的表述更加紧凑。引入定义的一个更实际的原因是:如果没有定义,逻辑或数学的文本会迅速增长而超过合理的范围。
		
		还有一种引入新名字的情况是引入变量,如“令 x 为一个实数,Let x be a real number”,两者的区别在于变量作为一个集合中任意一个实体的名字,而定义的名字则作为一个确定是东西或概念的名字。
\end{document}
