\documentclass[UTF8]{article}
\usepackage{ctex}
\usepackage{ulem}
\usepackage{amssymb}
\usepackage{amsmath}
\usepackage{graphicx}
\newtheorem{thm}{定义}[section]
\newtheorem{notation}[thm]{记号}
\newtheorem{lemma}[thm]{引理}

\makeatletter
\newcommand{\rmnum}[1]{\romannumeral #1}
\newcommand{\Rmnum}[1]{\expandafter\@slowromancap\romannumeral #1@}
\makeatother

\title{构造演算(The Calculus of Constructions)\\[2ex]\begin{large}读书笔记\end{large}}
\author{许博}
\date{}

\begin{document}
\maketitle
	\section{$\lambda{\rm C}$系统}
		$\lambda{\rm C}$组合了第二章到第五章中介绍的系统,拥有四种可能的选择,即依赖于项/类型的项/类型。
		
		$\lambda{\rm P}$与$\lambda{\rm C}$只有一处不同,但足以扩展$\lambda{\rm P}$到$\lambda{\rm C} = \lambda{2} + \lambda{\uline{\omega}} + \lambda{\rm P}$:
		
		$(form_{\lambda{\rm P}})\ \cfrac{\Gamma\vdash A:*\ \ \ \Gamma,x:A\vdash B:s}{\Gamma\vdash\Pi x:A.B:s}$\\
		
		在这条规则中,关键点是$A:*$,为了保证类型$\Pi x:A.B$的成员(inhabitant)是项或者依赖于项的类型。但在舍弃了这个限制之后,我们就获得了我们想要的泛化:依赖于项/类型的项/类型。
		
		看起来将$A:*$替换为$A:s$,其中$s$为$*$或$\square$,就足够了,但是规则中已经出现了$s$,观察$\lambda{\uline{\omega}}$的($form$)-规则:
		
		$(form_{\lambda{\uline{\omega}}})\ \cfrac{\Gamma\vdash A:s\ \ \ \Gamma\vdash B:s}{\Gamma\vdash A\rightarrow B:s}$\\
		
		只能表示依赖于项的项和依赖于类型的类型,而不能相互交叉(cross-over)。
		
		因此在$\lambda{\rm C}$的($from$)-规则中,使用了两个$s$:$s_1$和$s_2$:
		
		$(form_{\lambda{\rm C}})\ \cfrac{\Gamma\vdash A:s_1\ \ \ \Gamma,x:A\vdash B:s_2}{\Gamma\vdash\Pi x:A.B:s_2}$\\
		
		$\Pi x:A.B$的类型继承自$B$,也即依赖于项/类型(1)的项/类型(2)依然是项/类型(与2统一)。因此有一个有趣的事实是:假设$A$中不存在与抽象的类型变量相同的自由类型变量,$*\rightarrow A$是一个类型,而$A\rightarrow *$是一个种类(kind)。
\end{document}
