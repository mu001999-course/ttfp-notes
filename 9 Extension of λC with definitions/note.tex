\documentclass[UTF8]{article}
\usepackage{ctex}
\usepackage{ulem}
\usepackage{amssymb}
\usepackage{amsmath}
\usepackage{graphicx}
\newtheorem{thm}{定义}[section]
\newtheorem{notation}[thm]{记号}
\newtheorem{lemma}[thm]{引理}

\makeatletter
\newcommand{\rmnum}[1]{\romannumeral #1}
\newcommand{\Rmnum}[1]{\expandafter\@slowromancap\romannumeral #1@}
\makeatother

\title{9 以定义扩展$\lambda{\rm C}$\\Extension of $\lambda{\rm C}$ with definitions\\[2ex]\begin{large}读书笔记\end{large}}
\author{许博}
\date{}

\begin{document}
\maketitle
	\section{$\lambda{\rm C}$扩展到系统$\lambda{\rm D_0}$}
	\noindent
	本章在$\lambda{\rm C}$的基础上扩展通常意义上定义的形式化版本,也即所谓的描述性定义(descriptive definitions)。扩展后的系统$\lambda{\rm D_0}$尚不能完全支持公理以及公理概念的表示,相应的扩展会在下一章引入$\lambda{\rm D}$时说明。
	
		为给出$\lambda{\rm D_0}$的合适的描述,首先扩展表达式的集合。$\lambda{\rm D_0}$中的表达式与$\lambda{\rm D}$中相同,因此记集合为$\mathcal{E}_{\lambda{D}}$。
		
		假设除了之前定义的变量集合$V$以外,还有常量的集合$C$。使用符号$a,a_1,a_i,a',b,...$作为常量的名字,正如我们使用$x,x_1,x_i,x',y,...$作为变量的名字一样。另外,还假设变量和常量来自不相交的集合,而$*$和$\square$是特殊符号,不属于$V$和$C$:
		
		$V\cap C=\emptyset,\ *\not=\square,\ *,\square\notin V\cup C$
		
		\begin{thm}($\mathcal{E}_{\lambda{D}}$)
			
			$\mathcal{E}_{\lambda{D}}=V|\square|*|(\mathcal{E}_{\lambda{D}}\mathcal{E}_{\lambda{D}})|(\lambda V:\mathcal{E}_{\lambda{D}}.\mathcal{E}_{\lambda{D}})|(\Pi V:\mathcal{E}_{\lambda{D}}.\mathcal{E}_{\lambda{D}})|C(\overline{\}})$
		\end{thm}
	
		其中$\overline{\}}$中的上划线表示这是一个$\}$-表达式的列表。
		
		引入“环境,environment”表示一个定义的列表。
		
		\begin{thm}($\lambda{\rm D_0}$中的描述性定义;环境)
			
			\noindent
			(1) 在$\mathcal{E}_{\lambda{D}}$中,一个(描述性)定义具有形式
			
			$\overline{x}:\overline{A}\triangleright a(\overline{x}):=M:N$
			
			\noindent
			其中所有的$x_i\in V,a\in C$,并且所有的$A_i,M,N\in\mathcal{E}_{\lambda{D}}$
			
			\noindent
			(2) 一个环境$\Delta$是一个有限(空或非空)的定义列表。
		\end{thm}
	
		使用诸如$\mathcal{D},\mathcal{D}_i,...$等符号作为元名称表示定义。一个长度为$k$的环境可以被表示为如$\Delta\equiv\mathcal{D}_1,...,\mathcal{D}_k$。
		
		关于定义,区分以下元素:
		
		\begin{thm}(定义中的元素)
			
			\noindent
			令$\mathcal{D}\equiv\overline{x}:\overline{A}\triangleright a(\overline{x}):=M:N$是一个定义。则:\\
			- $\overline{x}:\overline{A}$是$\mathcal{D}$中的上下文\\
			- $a$是$\mathcal{D}$中被定义的常量,$\overline{x}$是参数列表\\
			- $a(\overline{x})$是$\mathcal{D}$中的$definiendum$\\
			- $M:N$是$\mathcal{D}$中的语句,$M$是$definiens$或$\mathcal{D}$的主体,$N$是$\mathcal{D}$的类型。
		\end{thm}

	\section{以定义扩展推定}
	\noindent
	回顾$\lambda{\rm C}$中的推定,具有如下形式:
	
		$\Gamma\vdash M:N$
		
		但在$\lambda{\rm D_0}$中,这样的一个推定可能会依赖一些定义,因此我们在推定之前添加环境,使用元符号“;”分割环境与推定,因此包含定义的推定具有新的形式:
		
		\begin{thm}(包含定义的推定;扩展后的推定)
			
			$\Delta;\Gamma\vdash M:N$,\\
			其中$\Delta$是一个环境,$\Gamma$是一个上下文以及$M,N\in\mathcal{E}_{\lambda{D}}$。
		\end{thm}
	
		其含义为:“在环境$\Delta$和上下文$\Gamma$中,$M$具有类型$N$”。
		
		因此$M:N$由在其头部的列表$\Delta$和$\Gamma$修饰:\\
		(1) 环境$\Delta$绑定了$M:N$中出现的常量,\\
		(2) 上下文$\Gamma$绑定了$M:N$中出现的自由变量。
		
		在整个推定中,存在依赖关系,先出现的变量或常量可能会出现在之后出现的部分中,而后出现的变量或常量则不会出现在之前出现的部分中,尽管前后可能存在相同的名称,但并非表示的不同。
		
		与上下文相同,使用$\Delta,\mathcal{D}$表示在$\Delta$右边以$\mathcal{D}$进行扩展。
		
		因为暂且不考虑递归定义,因此在一个定义当中,被定义的常量只出现一次。
		
		再给出修改后的全部推到规则之前,将先引入规则($def$)和($inst$),前者导入新的定义到已存在的环境中,而后者则是定义的实例化规则。
		
	\section{用于添加定义的规则}
	\noindent
	首先,描述如何扩展一个推定中的环境$\Delta$,它已被接受并且为正确的:
	
		(\rmnum{1}) $\Delta;\Gamma\vdash K:L$
		
		在其中添加一个新的并且良构的定义,需要保证添加的定义本身是良构的,考虑如下一个新的定义:
		
		$\mathcal{D}\equiv\overline{x}:\overline{A}\triangleright a(\overline{x}):=M:N$\\
		期望将其添加至$\Delta$的尾部。
		
		因为$\Delta$中定义的常量,可能会出现在$\mathcal{D}$中,为了使$\mathcal{D}$是可接受的,需要$M:N$在上下文$\overline{x}:\overline{A}$以及环境$\Delta$中是可推导的。
		
		因此我们需要一个条件:
		
		(\rmnum{2}) $\Delta;\overline{x}:\overline{A}\vdash M:N$
		
		从而我们得到了规则(def):
		
		\begin{thm}(用于添加一个定义到一个环境中的推到规则)
			
			\noindent
			令$a$是一个未在$\Delta$中定义的新名字,且$\mathcal{D}\equiv\overline{x}:\overline{A}\triangleright a(\overline{x}):=M:N$
			
			(def) $\cfrac{\Delta;\Gamma\vdash K:L\ \ \ \Delta;\overline{x}:\overline{A}\vdash M:N}{\Delta,\mathcal{D};\Gamma\vdash K:L}$
		\end{thm}

	\section{用于实例化定义的规则}
	\noindent
	在实例化定义时,实例化一个参数可能会改变声明列表中的类型,因为定义中的上下文中后面出现的类型,可能依赖之前的声明,考虑一个具有如下形式的定义:
	
		$\mathcal{D}\equiv x_1:A_1,...,x_n:A_n\triangleright a(x_1,...,x_n):=M:N$
		
		对于每个变量$x_i$,使用表达式$U_i$进行实例化。对于$U_1$,实例化$x_1$时,需要满足条件$U_1:A_1$。而对于$U_2$,因为$A_2$可能依赖$x_1$,所以需要进行替换,因此$U_2$需要满足条件$U_2:A_2\left[x_1:=U_1\right]$。因此$U_3:A_3\left[x_1:=U_1,x_2:=U_2\right]$,需要注意的是,因为$x_1$到$x_n$都是$\mathcal{D}$中上下文的变量,因此$x_i$并不出现在$\mathcal{D}$之外,所以替换时同时替换还是顺序替换并不影响替换的结果。
		
		以此类推,对于$U_i$,需要满足条件$U_i:A_i\left[x_1:=U_1,...,x_{i-1}:=U_{i-1}\right]$。因为$x_i,...,x_n$不会出现在$A_i$中,所以对于每个表达式$U_i$所需条件的通用格式为:
		
		$U_i:A_i\left[x_1:=U_1,...,x_n:=U_n\right]$
		
		套用之前的缩写形式,使用$\left[\overline{x}:=\overline{U}\right]$作为替换$\left[x_1:=U_1,...,x_n:=U_n\right]$的缩写。因此对于每个$U_i$,应有$U_i:A_i\left[\overline{x}:=\overline{U}\right]$。再次使用上划线,以$\Delta;\Gamma\vdash\overline{U}:\overline{V}$作为列表$\Delta;\Gamma\vdash U_1:V_1,...,\Delta;\Gamma\vdash U_n:V_n$的缩写,此时规则的$\bf premisses$被缩写为$\Delta;\Gamma\vdash\overline{U}:\overline{A\left[\overline{x}:=\overline{U}\right]}$。
		
		而因为$a(\overline{x}):N$,因此$a(\overline{U}):N\left[\overline{x}:=\overline{U}\right]$,因此可以得到实例化规则(的一部分,因为不包括无参数的定义实例化):
		
		\begin{thm}(用于实例化的规则,1)
			
			\noindent
			令$a$为一个没有参数列表的常量,令$\mathcal{D}\in\Delta$,其中$\mathcal{D}\equiv\overline{x}:\overline{A}\triangleright a(\overline{x}):=M:N$,则:
			
			(inst-pos) $\cfrac{\Delta;\Gamma\vdash\overline{U}:\overline{A\left[\overline{x}:=\overline{U}\right]}}{\Delta;\Gamma\vdash a(\overline{U}):N\left[\overline{x}:=\overline{U}\right]}$
		\end{thm}
	
		其中 pos 意为 positive,正数。
		
		对于没有参数列表的定义,$\bf premisses$ 会为空,此时无法保证 $\bf conclusion$ 中$\Delta;\Gamma$的良构与否(前文中规定由$\bf premisses$保证)。故此时添加一条简单的$\bf premiss$,以保证$\Delta;\Gamma$是良构的:
		
		\begin{thm}(用于实例化的推导规则,2)
			
			\noindent
			令$a$为无参数列表的常量,令$\mathcal{D}\in\Delta$,其中$\mathcal{D}\equiv\emptyset\triangleright a():=M:N$,则:
			
			(inst-zero) $\cfrac{\Delta;\Gamma\vdash*:\square}{\Delta;\Gamma\vdash a():N}$
		\end{thm}
	
		而将规则(inst-pos)和(inst-zero)结合便得到了规则(inst)以覆盖参数列表空和非空时的情况:
		
		\begin{thm}(用于实例化的推导规则)
			
			\noindent
			令$a$为常量,令$\mathcal{D}\in\Delta$,其中$\mathcal{D}\equiv\overline{x}:\overline{A}\triangleright a(\overline{x}):=M:N$,则:
			
			(inst) $\cfrac{\Delta;\Gamma\vdash*:\square\ \ \ \Delta;\Gamma\vdash\overline{U}:\overline{A\left[\overline{x}:=\overline{U}\right]}}{\Delta;\Gamma\vdash a(\overline{U}):N\left[\overline{x}:=\overline{U}\right]}$
		\end{thm}

	\section{定义展开以及$\delta$-变换($\delta$-conversion)}
	\noindent
	关于定义有一个重要的方面还没有形式化:使用被定义的常量表示定义的实体。也即需要形式化在定义了一个常量之后,常量和实体之间需要可以互相替换。
		
		与$\beta$-规约类似,我们首先引入单步定义展开或称$\delta$-规约,需要注意的是,这个展开始终关联于定义所出现的环境$\Delta$。
		
		\begin{thm}(单步定义展开;单步$\delta$-规约,$\stackrel{\Delta}{\rightarrow}$)\\
			若$\Gamma\triangleright a(\overline{x}):=M:N$是环境$\Delta$中的一个元素,则:\\
			(1) (Basis) $a(\overline{U})\stackrel{\Delta}{\rightarrow}M\left[\overline{x}:=\overline{U}\right]$\\
			(2) (Compatibility) 若$M\stackrel{\Delta}{\rightarrow}M'$,则$ML\stackrel{\Delta}{\rightarrow}M'L,LM\stackrel{\Delta}{\rightarrow}LM',\lambda x.M\stackrel{\Delta}{\rightarrow}\lambda x.M'$以及$b(...,M,...)\stackrel{\Delta}{\rightarrow}b(...,M',...)$
		\end{thm}
	
		其中 Compatibility (2) 将是 (1) 扩展到子表达式。以及符号$\stackrel{\Delta}{\rightarrow}$简写了$\stackrel{\Delta}{\rightarrow}_\beta$。$\delta$-规约与$\beta$-规约的区别在于,每次$\delta$-规约只替换一个定义的一次出现,而$\beta$-规约则替换所有相同的绑定变量。
		
		$\stackrel{\Delta}{\rightarrow}$的逆关系被称作(单步)折叠(folding):若$M\stackrel{\Delta}{\rightarrow}M'$,则$M$是在$M'$中折叠一个确定的实例化实体的结果。
		
		与$\rightarrow_\beta$扩展到$\twoheadrightarrow_\beta,=_\beta$相似的,也定义了概念“关联$\Delta$的零或多步$\delta$-规约,$\stackrel{\Delta}{\twoheadrightarrow}$”和“关联$\Delta$的$\delta$-变换,$\stackrel{\Delta}{=}$”:
		
		\begin{thm}($\delta$-规约(零或多步),$\stackrel{\Delta}{\twoheadrightarrow}$)\\
			记$M\stackrel{\Delta}{\twoheadrightarrow}N$,若存在$n$个表达式$M_0$到$M_n$,且$M_0\equiv M$,$M_n\equiv N$以及对于所有$0\le i\lg n$有$M_i\stackrel{\Delta}{\rightarrow}M_{i+1}$。
		\end{thm}
	
		\begin{thm}($\delta$-变换,$\stackrel{\Delta}{=}$)\\
			记$M\stackrel{\Delta}{=}N$,若存$n$个表达式$M_0$到$M_n$,且$M_0\equiv M$,$M_n\equiv N$以及对于所有$0\le i\lg n$有$M_i\stackrel{\Delta}{\rightarrow}M_{i+1}$或$M_{i+1}\stackrel{\Delta}{\rightarrow}M_i$。
		\end{thm}
	
		因此,$M$和$N$是可$\delta$-变换的,当其中一个可以由另一个经过若干次的定义展开或折叠时。以及,关系$\stackrel{\Delta}{=}$具有自反性(reflexive),对称性(symmetric)和传递性(transitive)。
		
		同样,与$\beta$-规约类似,定义一个表达式的$\delta$-标准形式($\delta-normal form$),关于一个环境$\Delta$:
		
		\begin{thm}(可展开的(unfoldable),$\delta$-标准形式,$\delta$-nf)\\
			令$\Delta$是一个环境\\
			(1) 如果一个常量$a$被约束到$\Delta$中的一个描述性定义,则$a$是关于$\Delta$可展开的(unfoldable)。\\
			(2) 如果$K$中不存在关于$\Delta$可展开的常量,则$K$满足关于$\Delta$的标准形式。\\
			(3) 如果存在满足关于$\Delta$的标准形式的$L$使得$K\stackrel{\Delta}{=}L$,则$K$具有一个关于$\Delta$的标准形式。也可以说$K$是可$\delta$-标准化的,以及$L$是$K$的$\delta$-标准形式。
		\end{thm}
\end{document}
