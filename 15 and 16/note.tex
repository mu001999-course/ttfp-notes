\documentclass[UTF8]{article}
\usepackage{ctex}
\usepackage{ulem}
\usepackage{dsfont}
\usepackage{amssymb}
\usepackage{amsmath}
\usepackage{graphicx}
\newtheorem{thm}{定义}[section]
\newtheorem{notation}[thm]{记号}
\newtheorem{lemma}[thm]{引理}

\makeatletter
\newcommand{\rmnum}[1]{\romannumeral #1}
\newcommand{\Rmnum}[1]{\expandafter\@slowromancap\romannumeral #1@}
\makeatother
\newcommand{\dperp}{\perp\!\!\!\perp}

\title{15/16章读书笔记}
\author{许博}
\date{}

\begin{document}
\maketitle

\newpage
\section{15 An Elaborated Example}
	证明略
	
\newpage
\section{16 Further Perspectives}
	\subsection{$\lambda{\rm D}$的应用}
	\noindent
	总结类型理论(尤其是$\lambda{\rm D}$)作为一个用于形式化数学的系统时的主要特性:
	
		\textit{通过类型理论的数学形式化,formalisation of mathematics via type theory}
		
		\textit{数学的检查,checking of mathematics} 可以检查不完备的证明或者使用了不合法的逻辑步骤的证明等。
		
		\textit{证明发展,proof development} 可以构建推理的步骤,也即证明的逐步发展,对于开始学习逻辑和数学的学生尤其有帮助。
		
		\textit{库,libraries} 通过命名定义与证明,可以得到一个巨大的环境,也即包含了数学概念和定理的定义的形式化的数学的库。
	
	\subsection{基于类型理论的证明助手}
	\noindent
	类型理论不止可以用于形式化数学,还可以用于编写计算机程序,作为证明助手(proof assistant),交互式地构建定理和证明,最终得到形式化和计算机检查的数学的一个计算机支持的库:
	
		\textit{计算机检查的证明,computer-checked proofs} 通过检查证明的良构与否以及是否可以得到它的类型,来实现通过计算机检查证明。
		
		\textit{交互式证明,interactive proving} 由一个上下文中的一个类型开始,交互式地构建出符合该类型的项,来实现交互式证明。
		
		\textit{自动化与策略,automation and tactics} 在基于类型理论的证明助手中,最重要的可能是使用系统构建项。比如,Coq系统具有强大的策略以构建证明项,比如\textit{intros}等。
		
		\textit{技术援助,technical assistance} 提供“搜索”机制,可以寻找关于给定关系的引理,或者给定形状的引理。
		
		\textit{额外的类型理论的特性,additional type-theoretic features} 比如 Coq 具有归纳类型(inductive types)。
		
	\subsection{领域的未来}
		\textit{增加证明助手的使用,increasing use of proof assistants}
		
		\textit{自动化,automation}
		
		\textit{形式化证明的高级解释,high-level explanation of formal proofs}
		
		\textit{循序渐进的证明发展,step-wise proof development}
		
		\textit{证明助手间的导出,export between proof assistants} 不同证明助手间导出结果,以复用。
		
		\textit{教学,didactics} 用于学生学习逻辑与数学。
\end{document}
